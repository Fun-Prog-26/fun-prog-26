\documentclass{article}

\usepackage{enumitem}
\usepackage{fancyhdr}
\usepackage{listings}
\usepackage{graphicx}
\usepackage[printsolution=true]{exercises}
%% Change this for title information 
\newcommand\ExTitle{State Monads}

\newcommand\fullExTitle{exercises \\ \ExTitle }
\newcommand\footerExTitle{\ExTitle -\  exercises }

\pagestyle{fancy}
\fancyhead{} % clear all header fields
\renewcommand{\headrulewidth}{0pt} % no line in header area
\fancyfoot{} % clear all footer fields
\fancyfoot[LE,RO]{\thepage}           % page number in "outer" position of footer line
\fancyfoot[RE,LO]{\footerExTitle} % other info in "inner" position of footer line

%\usepackage[mathrm,colour,cntbysection]{czt}

\begin{document}
\begin{Huge}
	\begin{center}
	\fullExTitle
	\end{center}
\end{Huge}

\begin{exercise}
Give the example of a stack in slide deck, write a similar program to manage a Queue structure. A Queue can be seen as a List of elements. The main functions of a Queue are 
\begin{enumerate}
  \item \textit{\textbf{enQueue}} (an element is added to the end of the list - similar to \textit{\textbf{push}} in stack)
  \item \textit{\textbf{deQueue}} (an element is returned from the top of the list - similar to \textit{\textbf{pop}} in stack)
\end{enumerate}
Write a short program to test the operation of the queue (similar to the example for the stack)

\end{exercise}

\begin{solution}
\begin{lstlisting}[language=Haskell]-- Define a Queue type
	type Queue a = [a]
	
	-- Add an element to the end of the queue
	enQueue :: a -> Queue a -> Queue a
	enQueue x q = q ++ [x]
	
	-- Remove an element from the front of the queue
	deQueue :: Queue a -> (Maybe a, Queue a)
	deQueue []     = (Nothing, [])         -- Empty queue case
	deQueue (x:xs) = (Just x, xs)
	
	-- Function to display the queue
	printQueue :: Show a => Queue a -> IO ()
	printQueue q = putStrLn $ "Queue: " ++ show q
	
	-- Test program
	main :: IO ()
	main = do
		-- Start with an empty queue
		let q0 = [] :: Queue Int
		printQueue q0
	
		-- Enqueue some elements
		let q1 = enQueue 10 q0
		let q2 = enQueue 20 q1
		let q3 = enQueue 30 q2
		printQueue q3
	
		-- Dequeue an element
		let (e1, q4) = deQueue q3
		putStrLn $ "Dequeued: " ++ show e1
		printQueue q4
	
		-- Dequeue another element
		let (e2, q5) = deQueue q4
		putStrLn $ "Dequeued: " ++ show e2
		printQueue q5
\end{lstlisting}
\end{solution}	



\end{document}