\documentclass{article}
\usepackage{enumitem}
\usepackage{fancyhdr}
\usepackage{listings}
\usepackage{graphicx}
\usepackage[printsolution=true]{exercises} % put in false to hide solutions, true to print
% for exercise 
%% Change this for title information 
\newcommand\ExTitle{Topic 9 - Interactive Programming}

\newcommand\fullExTitle{Exercises \\ \ExTitle }
\newcommand\footerExTitle{\ExTitle -\  Exercises }

\pagestyle{fancy}
\fancyhead{} % clear all header fields
\renewcommand{\headrulewidth}{0pt} % no line in header area
\fancyfoot{} % clear all footer fields
\fancyfoot[LE,RO]{\thepage}           % page number in "outer" position of footer line
\fancyfoot[RE,LO]{\footerExTitle} % other info in "inner" position of footer line



\begin{document}
\begin{Huge}
	\begin{center}
	\fullExTitle
	\end{center}
\end{Huge}
\begin{exercise}
Write an I/O program which will read a line of input and test whether the input is a palindrome. The program should 'prompt' the user for its input and also output an appropriate message.
\end{exercise}

\begin{solution}
\begin{lstlisting}[language=Haskell]
interactivePalCheck :: IO ()

interactivePalCheck
  = do putStr "Input a string for palindrome check: "
       st <- getLine
       if st == reverse st 
          then putStr "Palindrome.\n"
          else putStr "Not a palindrome.\n"
\end{lstlisting}
          
\end{solution}

\begin{exercise}
Write an I/O program which will read two integers, each on a separate line and output their sum. The program should prompt for input and explain its output.
\end{exercise}

\begin{solution}
\begin{lstlisting}[language=Haskell]
interactiveIntSum :: IO ()

interactiveIntSum
  = do putStr "Input an integer (followed by Return): "
       st1 <- getLine
       let int1 = (read st1) :: Int
       putStr "Input another integer (followed by Return): "
       st2 <- getLine
       let int2 = read st2 :: Int
       putStrLn ("The sum of these integers is "++ show (int1+int2))
\end{lstlisting}
\end{solution}

\pagebreak
\begin{exercise}
Define a  function \\
\begin{lstlisting}[language=Haskell]
putNtimes :: Integer -> String -> IO ()
\end{lstlisting}
so that the effect of
\begin{lstlisting}[language=Haskell]
 putNtimes n str
\end{lstlisting}
 is to output a string \textit{str}, $n$ times, one per line. \\
 \textbf{\textit{Hint: }} You can use recursion in the definition. 
\end{exercise}

\begin{solution}
\begin{lstlisting}[language=Haskell]

putNtimes :: Integer -> String -> IO ()

putNtimes n st
  = if n<=0 
       then return ()
       else do putStrLn st
               putNtimes (n-1) st                                                   

\end{lstlisting}
\textbf{Alternatively }
\begin{lstlisting}
putNTimes :: Integer -> String -> IO ()
putNTimes 0 str  = return ()
putNTimes n str  = do putStr str
                      putNTimes (n-1) str
\end{lstlisting}
\end{solution}


\begin{exercise}
Write an I/O program which will first read a positive integer, $n$, and then read n integers and write their sum.  The program should prompt for input and explain its output.\\
\textbf{\textit{Hint: }} use auxillary functions, e.g.
\begin{lstlisting}[language=Haskell]
 getInteger :: String -> IO Integer 
 sumNInts ::     --- .... which sums n ints
\end{lstlisting}

\end{exercise}

\begin{solution}
\begin{lstlisting}[language=Haskell]
-- Instead of solving this as a single function, worth thinking about how you can 
-- decompose the problem: write a function to get an integer, and another
-- to do the summing.

-- Useful auxiliary function, taking the prompt as parameter.

getInteger :: String -> IO Integer

getInteger prompt
  = do putStr prompt
       st <- getLine
       return (read st :: Integer)

-- Sum N integers: prompt, number to sum and and "sum so far" are the parameters

sumNints :: String -> Integer -> Integer -> IO Integer

sumNints prompt n s
  = if n<=0 
       then return s
       else do m <- getInteger prompt
               sumNints prompt (n-1) (s+m)


--- The function itself

getNints :: IO ()

getNints
  = do bound <- getInteger "Input the number of integers to add: "
       sum <- sumNints "Input an integer: " bound 0
       putStrLn ("The sum of these integers is "++ show sum)
\end{lstlisting}
\end{solution}

\end{document}