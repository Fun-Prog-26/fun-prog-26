\documentclass{article}

\usepackage{enumitem}
\usepackage{fancyhdr}
\usepackage{listings}
\usepackage{graphicx}
\usepackage[printsolution=false]{exercises}
\usepackage{todonotes}  %--put in noanswer when solutions are not to be shown
\usepackage{color}
\definecolor{dkgreen}{rgb}{0,0.6,0}
\definecolor{gray}{rgb}{0.5,0.5,0.5}
\definecolor{mauve}{rgb}{0.58,0,0.82}


\lstset{frame=tb,
  language=Haskell,
  aboveskip=3mm,
  belowskip=3mm,
  showstringspaces=false,
  columns=flexible,
  basicstyle={\small\ttfamily},
  numbers=none,
  numberstyle=\tiny\color{gray},
  keywordstyle=\color{blue},
  commentstyle=\color{dkgreen},
  stringstyle=\color{mauve},
  breaklines=true,
  breakatwhitespace=true,
  tabsize=3
  }
%% Change this for title information 
\newcommand\ExTitle{Function Application and Function Composition}

\newcommand\fullExTitle{Exercises \\ \ExTitle }
\newcommand\footerExTitle{\ExTitle -\  Exercises }

\pagestyle{fancy}
\fancyhead{} % clear all header fields
\renewcommand{\headrulewidth}{0pt} % no line in header area
\fancyfoot{} % clear all footer fields
\fancyfoot[LE,RO]{\thepage}           % page number in "outer" position of footer line
\fancyfoot[RE,LO]{\footerExTitle} % other info in "inner" position of footer line

%\usepackage[mathrm,colour,cntbysection]{czt}

\begin{document}

\begin{Huge}
	\begin{center}
	\fullExTitle
	\end{center}
\end{Huge}

\begin{exercise}

Write \textbf{bigCubes} that takes a list and returns a list of cubes that are $>$ 500
\begin{lstlisting}
bigCubes :: [Int] -> [Int]
  \end{lstlisting}
\end{exercise}

\begin{solution}
  \begin{lstlisting}
bigCubes :: [Int] -> [Int]
bigCubes xs = filter (>500) $ map (^3) xs
  \end{lstlisting}
\end{solution}

\begin{exercise} 
  Write \textbf{lottaBiggest} that takes a list and replicates the largest element 4 times. 
  \begin{lstlisting}
  lottaBiggest :: [Int] -> [Int]
  e.g.
  lottaBiggest [2,5,3,1] = [5,5,5,5]
  \end{lstlisting}
  \textbf{Hint:} You can use the \textit{maximum} function that returns the maximum value in a numeric list. 
\end{exercise}

\begin{solution}
  Using \$  :
  \begin{lstlisting}   
lottaBiggest :: [Int] -> [Int]
lottaBiggest xs = replicate 4  $ maximum xs
      \end{lstlisting}

or (using function composition)
\begin{lstlisting}
lottaBiggest' :: [Int] -> [Int]
lottaBiggest'  = replicate 4  . maximum 
          \end{lstlisting}
\end{solution}


\begin{exercise} 
Write \textbf{powers} that takes a number and creates a list of that number squared, cubed, and quadrupled. 
\begin{lstlisting}[language=Haskell]
e.g. powers 2 = [4,8,16]
  \end{lstlisting}
\end{exercise} 

\begin{solution}
  \begin{lstlisting}[language=Haskell]

powers:: Int -> [Int]
powers x =  map ($x) [(^2),(^3) ,(^4) ] 
\end{lstlisting}
\end{solution}


\begin{exercise}
Given  \textbf{process} that takes a list of numbers, filters out the negative numbers, doubles the remaining numbers, and returns the sum of the doubled numbers:
\begin{lstlisting}[language=Haskell]
  process x = sum (map double (filter (>0) x))
\end{lstlisting}  
Rewrite \textbf{process} using function composition and eliminating the \textit{x} parameter.
\end{exercise}

\begin{solution}
\begin{lstlisting}[language=Haskell]
process :: [Int] -> Int
process = sum . map double . filter (>0)
\end{lstlisting}
\end{solution}

\pagebreak
\begin{exercise}
  Given 
  \begin{lstlisting}
  compute x y = length (take x (filter even y)) 
  \end{lstlisting}
  which takes a number and a list of numbers, filters out the even numbers, and returns the length of the first x even numbers.
  Rewrite \textbf{compute} using function composition and eliminating the \textit{y} parameter.
\end{exercise}

\begin{solution}
\begin{lstlisting}[language=Haskell]
compute :: Int -> [Int] -> Int
compute x = length . take x . filter even
\end{lstlisting}
\end{solution}

\begin{exercise}
Assume people are dining. We have a list of tip percents (assume people tip at different rates):\\
\begin{lstlisting}[language=Haskell]
e.g.  pcts = [0.15, 0.2, 0.21]  
\end{lstlisting} 
We have a list of bills (what people owe, minus tip)  \\
\begin{lstlisting}[language=Haskell]
e.g. amts = [20.5, 30, 25] 
\end{lstlisting} 
Write \textbf{calcBill }that takes amts and pcts and calculates what each person will pay, based on their amt and pct. 
Then apply a 4\% tax rate.

\begin{lstlisting}[language=Haskell]
calcBillamtspcts :: [Float] -> [Float] -> Float
calcBillamtspcts  [20.5, 30, 25] [0.15, 0.2, 0.21]   = [24.518,37.44,31.46]
\end{lstlisting} 
\end{exercise} 

\begin{solution}
\begin{lstlisting}[language=Haskell]
calcBillamtspcts :: [Float] -> [Float] -> [Float]
calcBillamtspcts amts tips =map (*1.04) $ zipWith (+) amts $ 
                                                      zipWith (*) amts  tips
   -- rightmost calculates tips , next adds tips, then add 4%
\end{lstlisting} 
\end{solution}




% \newpage
% \begin{Huge}
% \begin{center}
% Solutions
% \end{center}
% \end{Huge}
% \shipoutAnswer

\end{document}